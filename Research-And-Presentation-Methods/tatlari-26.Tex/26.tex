\documentclass{book}
\begin{document}
\begin{flushleft}
\texttt{CHAPTER TWO}
\hspace*{0.5cm}
\textbf{26}
\vspace*{0.7cm}
constructed by human beings. As such, finding useful and accurate information on the
Net requires a certain amount of skill as well as access to a variety of research and
retrieval engines.
\emph{\textbf{REFERENCES}}
December,J.(1994). Challenges for a webbed society. Computer-mediated Communication Magazine,
J(8).[online].Availabe:http://www.december.com/cmc/mag/1994/nov/websoc.html.
Diamond,E,& Bates,S.(1995).The ancient history of the Internet .American Heritage,46. 34-45
Jackson, M.(1997).Assessing the structure of communication on the World Wide Web.journal of
computer Mediated communication, 3(1).[online].Availabe:http://www.ascuse.org/jcmc/vol3/
issuel/jackson.html.
March, G.(1988).Hypermedia and learning:Freedom and chaos.Educational Technology,28(11),
8-12.
Nelson,T,H.(1967).getting it out of our system,In G.Schechter(Ed),\emph{Information retrival:A
critical review}(pp.191-210).Washington,DC:Thompsone.
Sterling,B.(1993).\emph{A short bistory of the Internet by Bruce Sterling}.[online].Available:http://w3
.aces.uiuc.edu/AIM/Scale/nethistory.html.
Under wired.(199).history of the /internet.[online].Available:http://www.underwired.com/report/
uw.css. 
\end{flushleft}
\end{document} 