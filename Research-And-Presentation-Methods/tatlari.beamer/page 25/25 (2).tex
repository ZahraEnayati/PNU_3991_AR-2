\documentclass[a4,9pt]{beamer}
\usetheme{Berlin}
\usepackage{multicol}
\usepackage{xcolor}
\usepackage{graphicx}
\linespread{1.35}
\usepackage{amsmath}
\usepackage{color}
\usepackage{tikz}
\usetikzlibrary{arrows,automata}
\begin{document}

\begin{frame}
\section*{Minimization of Finite Automata}
\begin{flushright}
 \texttt{WHAT IS NET?} \hspace*{0.1cm}\textbf{$|$} \hspace*{0.1cm} \textbf{25}\hspace*{0.1cm}
\end{flushright}
\vspace*{1cm}
Listen to sound segments and use proprietary meta tag system to selectively
Search through their large collections of songs and music.
An excellent and always available starting point for spelling, translation, the
saurus, and encyclopedia references, and occasional relevant links is the program
Atomica (http://www.atomica.ca). This free program, when installed on your
machine, provides instant (ALT CLICK) access to any word in your browser,
word processing document, or any other tex program very handy!
Other services function by quickly submitting your single request to multiple
search engines and displaying the resulte on a single(long!) screen.for example
DogPile (www.dogpile.com) “fetches” the results of your search from sixteen different
search engines.
Finally, most search engines allow the user to turn on a ”Kiddie filter” to eliminate
adulte content hits, and some such as SurfSafely (http://www.surfsafely.com/)
or Family Safe Startup Page (http://www.startup-page.com/) index only familyrated
sites.
\end{frame}
\begin{document}

\begin{frame}
\section*{Minimization of Finite Automata}
\begin{flushright}
 \texttt{WHAT IS NET?} \hspace*{0.1cm}\textbf{$|$} \hspace*{0.1cm} \textbf{25}\hspace*{0.1cm}
\end{flushright}
\vspace*{1cm}

Once you have decided which search tool to use, refer to that system’s online help Section.
Not all search engines work the same, nor do they have the have the same features and
commands. Moreover, there is nearly always advice and tutorials available that can be
extremely useful when refining your search, In the end, taking the time to read the
online help documentation will save you time by achieving better search result. Also,
keep in mind that rarely will you be satisfied with your frist search results. Be prepared
to try a variety of combinations of Boolean searches and synonyms for keywords.
finally, all search engines and subject guides have different methods of refining queries
and/or retrieving information. The best way to learn them is to read the help files on
the search engine sites, and don’t be afraid to experiment!
\end{frame}
\begin{document}

\begin{frame}
\section*{Minimization of Finite Automata}
\begin{flushright}
 \texttt{WHAT IS NET?} \hspace*{0.1cm}\textbf{$|$} \hspace*{0.1cm} \textbf{25}\hspace*{0.1cm}
\end{flushright}
\vspace*{1cm}
TIPS FOR FINDING USEFUL AND RELEVANT INFORMATION
being your search by analyzing your needs.
Isolate your keywords.
Select a search tool that matches your needs.
Experiment with a variety of search engines and subject guides!
Although our favorite search engines tend to change over time, we have found
Google(http://www.google.com) to be the fastest, most effective, and least adcluttred 
site.
SUMMARY
The Net has evolved into an unstructured network of millones of computers
through- out the world. Today most of us access the Net through the use of a
suite of protocols know as the
World Wide Web (WWW). Besides being described as the most com- plicated
network, the Internet has also been described as the largest network ever
\end{frame}
\end{document}